\documentclass{article}
\usepackage{tikz}
\usepackage{nopageno}
\usepackage[active,tightpage]{preview}  %generates a tightly fitting border around the work
\PreviewEnvironment{tikzpicture}
\setlength\PreviewBorder{2mm}

\begin{document}
\usetikzlibrary{decorations.pathreplacing}
\usetikzlibrary{arrows}

\begin{tikzpicture}[scale=1.0,>=angle 60]
% Visualization of the load aggregation of Claesson and Javed (2012)

% Highlight loads shifted to bigger cells
\fill[red] (5,10) -- (5,12) -- (6,12) -- (6,10) -- cycle;
\fill[red] (3,5) -- (3,6) -- (4,6) -- (4,5) -- cycle;
\fill[red] (5,5) -- (5,8) -- (6,8) -- (6,5) -- cycle;

% Generate axes
\foreach \y in {0,5,10}
{
    % x-axis
    \draw[line width=2pt,black,->] (0,\y) -- (10,\y);
    % y-axis
    \draw[line width=2pt,black,->] (8,\y) node[below right] {\large $t-t_i$} -- (8,\y+4) node[right] {\large Q};
    % y-axis ticks
    \foreach \dy in {1,2,3}
	\draw[line width=1pt,black] (8-0.15,\y+\dy) -- (8+0.15,\y+\dy);
}

% Cell delimiters
\foreach \n/\dt in {5/1,4/3,3/5,2/6,1/7}
	\draw[line width=1.5pt,black,dashed] (\dt,-0.5) -- (\dt,14) node[right] {\large \n};

% Identify cell widths
\foreach \dt in {2,4}
	\path (\dt,-0.5) node {\large $2\Delta t$};
\foreach \dt in {5.5,6.5,7.5}
	\path (\dt,-0.5) node {\large $\Delta t$};

% Loads in top graph
\foreach \dt/\ddt/\Q in {5/1/2,6/1/3,7/1/1}
	\draw[line width=1.5pt,blue] (\dt,10) -- (\dt,10+\Q) -- (\dt+\ddt,10+\Q) -- (\dt+\ddt,10);
% Loads in middle graph
\foreach \dt/\ddt/\Q in {3/2/1,5/1/3,6/1/1}
	\draw[line width=1.5pt,blue] (\dt,5) -- (\dt,5+\Q) -- (\dt+\ddt,5+\Q) -- (\dt+\ddt,5);
% Loads in bottom graph
\foreach \dt/\ddt/\Q in {1/2/0.5,3/2/2,5/1/1}
	\draw[line width=1.5pt,blue] (\dt,0) -- (\dt,0+\Q) -- (\dt+\ddt,0+\Q) -- (\dt+\ddt,0);

% Arrows
\draw[red,line width=2pt,->] (5.5,9.85) to [out=-90, in=90] (4,6.15);
\draw[red,line width=2pt,->] (3.5,4.85) to [out=-90, in=90] (2,0.65);
\draw[red,line width=2pt,->] (5.5,4.85) to [out=-90, in=90] (4,2.15);


\end{tikzpicture}

\end{document}